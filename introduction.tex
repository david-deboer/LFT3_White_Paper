The "Wow" signal seen in 1977 remains an tantalizing signal to this day.  6EQUJ5 seen once on a page.  It was so intriguing because the world didn't have so many transmitters on people, in the air or in the heavens above and it seemed to conform to the expected antenna beam.  Not conclusive since only seen the once and there were conceivable human-based explanations.  Imagine to go to a place where literally ``everything'' seen is of interest and to be able to explore with modern technology that dramatically exceeds what was on that 1977 telescope.  

Humanity is at a tipping point for conducting effective radio astronomical observations from Earth. The ever-increasing use of wireless communication devices on the ground and in orbit means that nowhere on our planet, even  remote locations with very low population density, is free of significant radio frequency interference (RFI). We have already gone to space with a small number of very expensive ``Great Observatories'', but the advent of accessible and relatively cost-effective access to space means that deploying more and smaller telescopes is now feasible. We are therefore at the tip of the spear of the transition from earth-based to space-based telescopes for radio astronomy --- born of necessity to respond to changes in Earth’s RF environment\footnote{Note that other wavelengths have always needed to be above the atmosphere due to absoprtion and scattering by atmospheric constituents.}; observations above the Earth's ionosphere also open up an observing window inaccesible from Earth.

While terrestrial wireless communications have strictly negative implications for radio astronomy, cost-effective access to space means that smaller telescopes and experiments may be deployed to areas with significantly less RFI. As this happens, the sensors themselves along with other active transmitters in space will begin injecting RFI into those domains. This makes it urgent to get well-designed sensors into space now to make early baseline measurements and conduct the unique science allowed by the current environment which benefits from a complete lack of RFI. The lunar farside is unique in our solar system in that it always has its back turned to the Earth \citep{heidmann2002,MACCONE2019233,michaud2020lunar}. The lunar farside currently presents a once-in-human-history opportunity to record signals in a fully quiet environment. 

To act on this singular opportunity, we propose a telescope to be landed near the lunar antipode within five years to conduct these unique-in-history measurements. The telescope will comprise of a UHF dual-pol multibeam phased array operating from $300-900$~MHz as well as individual elements covering $1-50$~MHz, $50-100$~MHz and $600-1800$~MHz. During its mission lifetime, the telescope will observe most of the lunar farside sky and conduct historical surveys in the most RFI-pristine environment in the solar system before humanities presence increases the radio frequency interference.