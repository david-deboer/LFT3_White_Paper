The "Wow" signal seen in 1977 (\citep{wow} remains the most tantalizing unexplained signal to this day.  %6EQUJ5 seen once on a page.  
It was so intriguing because the world didn't have so many transmitters on people (i.e. heart monitors and smart watches), in the air or in the heavens above, and it seemed to conform to the expected antenna beam. Due to the fact that it was only seen once, no conclusive evidence has been provided, and there was no plausible human explanation. Imagine going to a place where literally everything seen is of scientific interest, and being able to explore using modern technology that is far more advanced than the telescope used in 1977.   

Humanity is at a tipping point for conducting effective radio astronomical observations from Earth. The ever-increasing use of wireless communication devices on the ground and in orbit means that nowhere on our planet, even remote locations with very low population density, is free of significant radio frequency interference (RFI). We have already gone to space with a small number of very expensive ``Great Observatories'', but the advent of accessible and relatively cost-effective access to space means that deploying a large number smaller telescopes is now feasible. We are therefore at the tip of the spear of the transition from earth-based to space-based telescopes for radio astronomy --- born of necessity to respond to changes in Earth’s RF environment\footnote{Note that other wavelengths have always needed or at least prefered to be above the atmosphere due to absoprtion and scattering by atmospheric constituents.}; observations above the Earth's ionosphere also open up an observing window inaccesible from Earth.

While terrestrial wireless communications have strictly negative implications for radio astronomy, cost-effective access to space means that smaller telescopes and experiments may be deployed to areas with significantly less RFI. As this happens, the sensors and their associated electronics, along with other active transmitters in space will begin injecting RFI into those domains. This makes it urgent to get well-designed sensors into space now to make early baseline measurements and conduct the unique science allowed by the current environment which benefits from a complete lack of RFI. The lunar farside is unique in our solar system in that it always has its back turned to the Earth \citep{heidmann2002,MACCONE2019233,michaud2020lunar}. The lunar farside currently presents a once-in-human-history opportunity to record signals in a fully quiet environment. 

To act on this singular opportunity, we propose a telescope to be landed near the lunar antipode within five years to conduct these unique-in-history measurements. The telescope will comprise of a UHF dual-pol multibeam phased array operating from $300-900$~MHz as well as individual elements covering $1-50$~MHz, $50-100$~MHz and $600-1800$~MHz. During its mission lifetime, the telescope will observe most of the lunar farside sky and conduct historical surveys in the most RFI-pristine environment in the solar system before humanities presence increases the radio frequency interference.

Besides considerations for RFI, there are other scientific reasons for placing a radio telescope on the moon. The ionosphere surrounding our planet creates a conducting medium (or a dispersion) through which electromagnetic radiation must pass through. Although, this will not have a significant effect on observations above 1\,GHz, the lower frequencies are impacted as a wavelength squared relationship. Therefore, as we head toward meter wavelengths, we are impacted by shifting of the source position and potential for plasma screens making the sky opaque. Ground-based observations become infeasible after a frequency cutoff of 10--30\,MHz. The moon does not have an ionosphere and thus no barrier exists to study physics at these new frequencies, as outlined in the science section.

The main considerations when considering this endeavor are:
\begin{itemize}
    \item At low frequencies the system temperature will be dominated by Galactic noise and impulsive events will be affected by scattering.  This applies anywhere in the Galaxy so any civilization generating a beacon will be aware of this constraint.
    \item Having the widest possible absolute bandwidth will maximize the possibility for serendipitous discoveries.
    \item In the frequency band of 0.5--1\,GHz, there are numerous Earth-based telescopes that can carry out commensal or follow-up observations.
    \item The possibility of initial discovery of technosignatures in that band with Earth-based telescopes is drowned out by ``anthropogenic technosignatures” aka RFI.
    \item Lusee-Night is an existing proposal for a SZM lander operating at radio frequencies (0.1 to 50MHz) which are below the Earth's ionospheric cut-off to make radio astronomical observations not possible. 
    \item An all-sky monitor from the lunar surface will give an unprecendented view of bright, but rare transient events.
\end{itemize}

We note that an observatory in the far-side of the moon has no visibility of Earth, and thus would have limited use for defense/intelligence purposes, emphasizing its peaceful, scientific-use only purpose. Potential science cases for bringing this telescope to life are outlined in the following section.