The lunar farside provides a unique accessible environment in that the moon shields it from Earth and near-Earth RFI at all times; placing LFT3 within a crater provides additional protection. Therefore, {\em anything} detected is of scientific interest. Since there is no ionospheric blockage, low frequency observations that are impossible from Earth can also be made in an RFI-free environment. Although the lunar farside will remain an excellent low-RFI site for many years to come, {\em now} is the only time to take completely pristine measurements of the radio spectrum. This is an opportunity that will never again present itself.

\subsection{Technosignatures}
Humanity's quest to determine its place in the Universe and whether we are its only inhabitants has driven knowledge and imagination since we first looked to the sky.  Many large telescopes are in consideration in order to help determine if other biological processes are happening in nearby stars (called "biosignatures") and we now also have the capability to explore whether technological signatures (technosignatures) exist over a large swath of our Galaxy.  The Breakthrough Listen program has transformed the field and has greatly expanded the search for such technosignatures (e.g. \citealt{Enriquez_2017, Price_2020, Gajjar_2021}).  The greatest impediment is the prevalence of RFI made by essentially every device that is electrically powered.

Strong leakage (planetary radar interplanetary communication) from nearby...?
\begin{figure}
    \centering
    \includegraphics[width=0.75\linewidth]{figures/isaacsonetalcounts.png}
    \caption{Detection counts for the Breakthrough Listen target selection for nearby stars \citep{2017PASP..129e4501I} for an ``Arecibo'' scale transmitter (top) and 1000 times larger (bottom).  The black background is the total number per signal-to-noise ratio bin and the colors are the individual spectral types.  The first bin extends from 3-5 and the last bin is 25 and greater.}
    \label{fig:isaacsonetal}
\end{figure}

\begin{figure}
    \centering
    \includegraphics[width=0.75\linewidth]{figures/gaia_100lycounts.png}
    \caption{Detection counts for the Gaia Nearby Star Catalogue \citep{2021A&A...649A...6G} and for the catalog of spectral types within 100 ly (https://chview.nova.org/solcom/index.html for an ``Arecibo'' scale transmitter (top) and 1000 times larger (bottom).  The dark gray is the total number of the spectral type catalog stars per bin and the light gray is the total Gaia nearby star catalog; for the top plot they overlap.  The bins are the same as in \ref{fig:isaacsonetal}.}
    \label{fig:gaia100ly}
\end{figure}

Signals of interest must be carefully sought within this cacophony of RFI. To differentiate between RFI and a technosignature requires that the technosignature signal is at a frequency/time not inhabited by RFI and the signal must persist and be quite stable over a timeframe of many minutes.  The signals are assumed to be nearly monochromatic and slow-drifting in time.  A series of on-off measurements are used (for single-dish experiments)to make sure that the signal-of-interest is coming from the main beam of the telescope and not RFI in a distant sidelobe.  These techniques are incredibly powerful in conducting senstive searches over the available bandwidth, and have the added benefit being able to use very large earth-based telescopes to dramatically increase the sensitivity.  However, many limitations exist on appropriate ranges of frequency and time, and some frequency ranges are so impacted that searches (and radioastronomy research generally) are infeasible in those bands.  Signals of short duration (a few seconds) or signals heavily modulated are nearly impossible to discern.  And of course, technosignatures ``behind'' RFI is not detectable (or at least believably so).

With online beamforming capabilities we can search the nearest stars  across a range of frequencies rarely searched from Earth but represent some of our most power transmitters \citep{Tremblay_2022}. Within the current FPGA framework, calibration and dynamic spectrum creation can be completed giving us access to the full details of the signals detected for each frequency range. This combined with a sample of the time-averaged spectrum will provide a unique dataset in which to seek an answer to ``Are we alone?". Assuming a civilization is capable of building a big dish to generate an isotropically distributed signal of $\sim$10${^17}$\,W, with a 6$\sigma$ sensivity of 0.5\,Jy, we could expect to detect that signal out to about 3\,pc. However, a beamed signal or a large megastructure is expected to create power over 10$^{26}$\,W. 

The $Gaia$ Space telescope has given has a large catalog of targets which we are already using in blind, commensal surveys with MeerKAT (Czech et al. in prep) and the Karl G. Jansky Very Large Array \citep{Tremblay_2024}. From this experience we can develop and efficient search working on the FPGA framework.

\textbf{Science Specifications:} This science case would require some in-line processing of the data to find signals of interest and generation of dynamic spectra or raw voltage postage stamps around signals of interest. On the commensal, real-time system at MeerKAT and the VLA we use \textsc{seticore} to search for the signals toward beamformed targets. \textsc{seticore} using an efficient Taylor-Tree dedispersion algorith to autonomously search for narrowband signals. The user specifies the signal-to-noise ratio to identify signals of significance and the drift rate search width. To save on information needing to be transfered to the ground, instead of sending the dyanmic spectrum, we can de-disperse the signal and only transmit the time averaged power spectrum. 

As there is no know signal thus far identified, the frequency resolution and time resolution are open parameters that are flexble to needs to keep data rates and processing to a minimum. However, most of current radio astronomy experiments utilize $<$10\,Hz frequency resolution. 

We can verify these modes of operation by looking toward Mars where there are number of space-based and ground-based communication signals heading back to Earth on a regular basis. The carrier signals are often emitted at around 2\,GHz and represent a population of narrow-band artificial signals.

\subsection{Transients}
The recent flourishing of the field of radio transients (both slow and fast) has brought a wealth of information, significantly advancing various areas of physics. Notably, pulsars and fast radio bursts (FRBs) stand out due to their unique potential to reveal new kinds of physics. These astrophysical phenomena offer valuable insights into gravitational waves, cosmology, and plasma physics, among other fields. However, beamforming radio observations can also shed light on other transients such as flare stars and planetary conditions within and outside the solar system. Overall, the strenght of a lunar-based telescope is in discovering the unknown. This could include the possible detection of various “low-DM” events, on timescales of milliseconds to minutes: low-DM FRB analogues, FRBs from galactic magnetars, decimetric solar bursts, stellar bursts, and undiscovered time-domain phenomena. In particular, the discovery of new populations of transient astrophysical sources in the very local Universe.

\subsubsection{Pulsars}
Pulsars are highly magnetized, rotating neutron stars that emit beams of electromagnetic radiation. Pulsars serve as precise cosmic clocks, providing insights into the interstellar medium, gravitational waves, and the fundamental physics of matter under extreme conditions. It is theorized that pulsars spin rapidly upon creation and slow down due to gravity over time, along with changes in the medium surrounding the pulsar, which alters the pulse structure as well \citep{LW_2013}. By using coherent beamforming, sensitivity to weak pulses can be improved, and frequency and time resolution can be adjusted to decrease the effects of dispersion measure smearing (e.g. \cite{WL_2020}).\\

Broadband studies of pulsars allow for a study of the emission physics and surrounding material, where the lower frequencies are sensitive to emission closer to the pulsar surface. With ground-based radio telescopes, the ionosphere can create challenges, especially in weaker pulsars that have a peak spectral turn-over between 100--200,MHz \citep{Stappers_2011}. By moving the radio telescope off from the Earth and away from the ionosphere cut-off around 30,MHz, a broader range of pulsar emission physics can be studied than has yet been explored.\\

A farside lunar telescope, shielded from Earth’s radio frequency RFI, would significantly enhance pulsar observations by providing a pristine observational environment. This location allows for low-frequency studies, critical for understanding the early stages of pulsar emission mechanisms and the surrounding interstellar medium. The absence of atmospheric effects enables precise measurements, leading to better constraints on pulsar timing and more accurate tests of general relativity. Furthermore, the lunar farside offers an opportunity to continuously monitor pulsars without the interruptions caused by the Earth’s rotation, increasing the data quality and allowing for the detection of subtle changes in pulsar behavior over time.

**Provide a di-urnal systematic correction for pulsar timing arrays on Earth. Even doing a calibration of this “noise” factor on VELA may be enough.

\textbf{Science Specifications:} Some known pulsars are so bright and regular that even the simplest setups could detect them. However, for a blind survey of potentially irregular or rare pulsar emission within the inocoherent beam takes a bit more processing. 

There is a need for a folding mode for the pulsars and the mode can be tested on bright, known sources. This is to test our ability for precise time-keeping and can be verified with ground-based telescopes.



\subsubsection{Fast Radio Bursts}
Fast Radio Bursts (FRBs) are intense, millisecond-duration radio pulses originating from extragalactic sources. Their origins remain one of the most intriguing mysteries in modern astrophysics. Although Galactic magnetars have shown to exhibit similar FRB-like emission \citep{BC_2020}, the exact origins and emission mechanisms are still unknown, with a potential at the population of FRB signals is made of several sub population of sources.  \\ 

The Canadian Hydrogen Intensity Mapping Experiment (CHIME) telescope is currently the world leader for detecting new Fast Radio Bursts with its 200\,deg$^{2}$ field-of-view and observing frequency of 400--800\,MHz. %If we assume a fluence limit of 10\,Jy\,ms at 100\,MHz, and then we could detection 135 of the FRBs currently detected by CHIME. This means, assuming a similar amount of observing time on sky, we could detect $\sim$7 new FRBs per year with a lunar telescope.
%With the fluences and spectral index published in the CHIME FRB catalog, along with rate of detection, we can estimate the expected rate of new detections at 100\,MHz  \textcolor{red}{XXX} per year. 
Where a low-frequency Lunar telescope can provide a large benefit in understanding the true nature and populations of FRBs is through the study of emission at lower radio frequencies with high time resolution, as this is becoming a clear factor in understanding the source populations. \\ 

Currently, the Commensal Real-time ASKAP Fast Transients Collaboration (CRAFT; 700--1.8\,GHz) offer the largest number of localized repeater sources \citep{SM_2023}, with some more recent contributions from the Five-hundred-meter Aperture Spherical Telescope (FAST) collaboration (1.00--1.45\,GHz\citep{ZX_2023}) and the DSA-110 \citep{SD_2023,LC_2023}. The missing gap in understanding the true nature and originating source of emission is observing away from the influence of RFI and away from the ionospheres influences on the signal dispersion. \\ 

Low-frequency observations of FRBs are crucial for understanding their propagation through the intergalactic medium, potentially unveiling the distribution of baryonic matter and offering clues about the mechanisms driving these bursts. A farside lunar telescope, free from RFI and ionospheric distortion, would enable unprecedented low-frequency observations, shedding light on FRB origins, propagation effects, and contributing significantly to the broader understanding of cosmic phenomena.

\textbf{Scientific Specifications:} Given the values shown in Figures \ref{sensitivity} and \ref{sefd}, we calculated the expected sensitivity to FRB emission. If we assume 20\,seconds of integrated time for a de-dispersed signal at a bandwidth (BW) of 100\,MHz, then we would have a 5\,$\sigma$ flux density limit of 10--11\,Jy over an incoherent sum. The CHIME telescope evaluated 25 repeating FRBs and found them to have 63\,ms of peak widths with bandwidths of 50-200\,MHz \citep{CHIME_RepeatingFRB}. Therefore, we would need a time resolution of $<$1\,ms. However, they found the dispersion measures (DMs) to range from 220 to 17,000\,pc\,cm$^{-3}$ meaning we would need the compute capacity to run through this range to identify blind candidates. 

Most sources had the strongest emission at 400--600\,MHz, suggesting we would only need to do this experiment with one set of dipoles. The CHIME experiments have a sensivity limit of 0.339\,Jy, suggesting that with a limit of 11\,Jy it would be challenging to find new sources of emission from extragalactic-FRBs. An increase in bandwidth or time would add a $\sqrt{BW\times\tau}$ increase in sensitivity, where $\tau$ is the total integration time in seconds and BW is in Hz. However, there may be an undiscovered poplulation of galactic-FRB analgoues with low DMs, allowing for less compute. In either scenario, a de-dispersed spectrum would be the most compressed output for sending to Earth for further analysis.

\subsubsection{Flare Stars}

Radio flare stars, often young, magnetically active stars, provide insight into stellar magnetic activity, star formation, and the early stages of stellar evolution. Observations from a farside lunar telescope could reveal details about the mechanisms driving these flares and their impact on the surrounding stellar environment. Studying radio flare stars also has implications for exoplanetary science. The intense radiation from these flares can affect the habitability of orbiting exoplanets. A farside lunar telescope lends it self to assess the radiation environment of these planets, contributing to our understanding of exoplanet habitability based on the low frequency window observed and the unique RFI enviroment the farside has. \\ 

Observing these stars, especially at low frequency $(\leq 300 \ \text{MHz})$\footnote{$\lambda \sim 1 \ \text{m}$} act as probes into stellar and planetary plasma environments. Coronal Mass Ejections (CMEs) have a low-frequency burst component in which information about the kinematics of plasma can be deduced \citep{villadsen_ultra-wideband_2019}. Incident solar wind is also the primary driving force of auroral emission on magnetized planets. Radio emission from stars is typically produced through CMEs, observed phenomenologically in the Sun as type II and III solar bursts. The radio emission from these events is thought to be produced through the electron-cyclotron maser instability \citep[ECMI;][]{EMI}. The ECMI is a plasma instability that occurs in the presence of a magnetic field and a population of energetic electrons. The instability is thought to be the primary driver of radio emission observed in type II bursts and stars as a whole. With type III bursts being thought to be the result of electron beams that are accelerated in the corona. \\

Emission from planets has also been observed in the form of auroral emission. The most notable example of this is the Jovian system. The Jovian system is known to have auroral emission that is driven by the interaction of the solar wind with the magnetosphere of the planet. Coherent radio emission from the aurora is thought to be produced through the cyclotron maser instability \citep[ECM;][]{zarka_auroral_1998} which injects a high-velocity electron population into the magnetosphere. The maximum frequency of the ECM is directly proportional to the magnetic field strength of the object at its emitting point \citep{kavanagh_hunting_2023, joe_nature_review}.\\

Much time has been spent observing Ultracool Dwarfs ($\text{M7} <$) as they provide a good analogue to the Jovian systems. This allows for direct comparison between the two.  This provides valuable insights into the formation, atmospheric processes, and potential habitability of gas giants like Jupiter. Moreover, such comparative studies contribute significantly to our understanding of planetary evolution and diversity beyond our immediate neighbourhood. Recent detection of radiation belts around a UCD further supports the analogy to Jupiter, as radiation belts are a key characteristic of Jupiter's magnetosphere \citep{joe_nature_review}. \ UCDs, being intermediate in mass between stars and planets, serve as a bridge to understanding exoplanet detection. The discovery of bursting radio emission from a brown dwarf and subsequent detections in UCDs indicate departures from established stellar coronal/flaring relationships. \ Radio bursts from UCDs can exhibit periodic timing \citep{hallinan_rotational_2006} along with strong circular polarization and high brightness temperatures, suggesting the involvement of the ECM process in generating radio emission. Despite a significant amount of gigahertz-frequency radio searches, detection rates for UCD radio emissions remain stubbornly low at around 10\% overall \citep{lynch_radio_2016}. Given that LFT3 will observe from ....????

\textbf{Scientific Specifications:}

% This depends if Uranus or Neptune fall in any of the beams throughout the observatories lifetime and what kinda tsys we get at those times. 
\subsubsection{Solar System Emission} 

There are numerous sources of radio emission throughout the solar system, such as the Sun, Jupiter and Saturn. These emissions are produce through various dynamic events such as solar flares, CMEs and planetrary magneto-spheric interactions. Noteably, Jupiter's strong magnetic field interacts with its moons that produce strong (\~ X mJy) radio emissions. During it's flyby, Voyager 2 discovered strong radio emission from Neptune suggestive that it had a strong magnetic field as well \citep{ZHANG199237}.\\ 
% We should mention Zarka's paper on Uranus lighting, the only time we've ever seen it! Possible science goal for this as well 
\textbf{Scientific Specifications:}

\section{Continuum \& Spectral Lines}

\subsection{Continuum Imaging}
**studies of the sun
**bright luminous galaxies at low frequencies
**map of the sky at the lowest radio frequencies

\subsection{Galactic HI}
Hydrogen is one of the most prominent elements in the Universe and the neutral hydrogen line at 21\,cm (1.420\,GHz) is one of the most prominent diagnostic lines used in radio astronomy. From testing models of the early Universe to tracing the spiral arms of our Galaxy, H{\sc I} is an important line for understanding motions and dynamics. However, on Earth, even though the spectral region around the 1.42\,GHz emission is protected and reserved for radio astronomy, the spectral region around the line is heavily polluted with RFI. This makes studying galaxies within our local Universe and beyond challenging. 

\textbf{Scientific Specifications:}

\subsection{Radio Recombination Lines}
The diffuse cold neutral medium (CNM; T$_{s}$ $<$ 100\,K) is an important component of the interstellar medium (ISM). To date this gas has been primarily studied using the HI 21 cm line in absorption \citep{Dickey_1990}. The low ionization potential of carbon (11.4\,eV) permits carbon atoms, in the form of cold radio recombination lines (CRRL), to be ionized by the far-UV radiation fields. A second powerful and complementary probe of the conditions in the neutral ISM is provided by the CRRLs that arise in this ionized gas. CRRLs at low radio frequencies ($<$ 1.5 GHz) have been detected in the Galactic plane in both emission and absorption with a number of telescopes (e.g. \citealt{Kantharia_2001,Salas_2019}). 

At large n-bound states, observed at frequencies less than 100 MHz, the relative populations of atomic levels are controlled primarily by collisional processes. This makes it the population levels close to the kinetic temperatures of less than 100\,K and so the lines are detected in absorption along the line of sight towards strong continuum sources. The population becomes inverted at lower n-bound states at frequencies greater than 200/,MHz, resulting in emission lines being detected \citep{Tremblay_2018}. In the range of 100 to 200\,MHz, the conversion will take place at a frequency and intensity dependent upon the temperature and density of the gas within the cloud. This means that the more dense the gas, the conversion will shift toward higher frequencies. This makes CRRLs excellent pressure and temperature probes of the ionized gas regions of the ISM \citep{Salas_2019}.

The ionosphere negatively impacts scientific measurements as a wavelength squared dependence, thus having a greater impact on lower frequency observations. Observing low-frequency ($<$500\,MHz) CRRLs is critical in understanding the temperatures and conditions of the ionized gas fronts in the cold gas \citep{Salas_2018}. By removing the telescope from the ionosphere, potential artifacts and features in the spectral data that are not real would be removed, as well as the possibility of artificial spectral broadening, which could change the measured conditions.

\textbf{Scientific Specifications:} The CRRLs will change in width depending on the frequency of detection and the conditions of the environment. However, a 0.5\,kHz resolution would be sufficient to study the lines across proposed frequency bands. Additionally, the data need to only have a 8--10\,second time resolution. Correlated visbilities would allow for a low resolution map of the regions where the signals are detected but beamforming can be used if either the location of the source is known or in an incoherent beam where the lines are particularly strong. Overall, spatial resolution of one degree on the sky would allow for follow-up by ground-based telescopes where more detail is required.

\section{Environmental Observation}

\subsection{Spectral Monitoring}
Although the lunar farside should be a pristine RF environment, there will be the potential of some RFI from the communication orbiters that will be in place, as well as spacecraft that are at distances further than the moon.  There is also the potential of unknown actors emitting radio frequencies.  During the LFT3 operational period there is also the potential of additional payloads being deployed, for example China's orbiting interferometer called Discovering the Sky at the Longest Wavelengths (DSL).  Measuring a legacy spectral baseline as well as the potential increasing emergence of RFI is a key objective.

\begin{figure}[h]
    \centering
    \includegraphics[width=0.6\textwidth]{figures/Science Case Plots/galactic_projection_NEA.png}
    \caption{Some plot here with target of intrests that fall within the zenith pointings and their associated Jy measurements based on Tsys etc. }
    \label{fig:enter-label}
\end{figure}


***Learn about lunar surface topography, conductivity, radiation, bounce back of radiation.\\
***Learn how to do astronomy from the moon and its challenges scientifically.  This could be important as we pollute the Earth's skies with our own technology.\\
***How long could the equipment last in the environment? What is the limiting factor for this?


%PLOTS That could be useful 
% Frequency/Wavelength against Flux Density of Transients -> parameter space plot 
% Variation of Tsys (basically mapping galactic forground) as a function of Lunar Latitude 
% Senstivity as function of integration time 
% Galactic latitude on the x axis, sensitivity on the y, carious longitudes overlapped. 
% Galactic projection plot for areas/exoplanets of intrest, i.e. benchmarking well known targets. 
% Also worth running some numbers to see how much time A team targets are in primary 