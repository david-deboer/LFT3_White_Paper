This paper presents a science case for an instrument landing and operating from the lunar farside surface.  Other possible missions obviously exist, which will be briefly discussed below.
However, we believe that a lunar farside surface instrument possesses compelling and unique aspects making it the preferred mission.

%CDT: Suggest moving this all to the indroduction

\subsection{Surface of Earth}
%The surface of the Earth obviously presents the uniquely beneficial property of ease of access. This ease of access of course also applies to nearly all of human activity that know produces radio frequency interference.  Additionally, the atmosphere impacts the transmission of radio waves at very low frequencies (below around 20 MHz) and high frequencies (above around 20 GHz, which generally don't suffer RFI issues).
Even though radio telescopes are often built in remote places away from dense urban areas, they are still significantly more accessible than space-based instruments. This proximity allows for large apertures, maintainable and changeable hardware, and the ability to process and transport large amounts of data. However, the disadvantages are two-fold and severe: 1) the atmosphere and ionosphere block and distort signals at some radio frequencies and 2) on- and off-world human activities cause significant Radio Frequency Interference (RFI) that results in severe artifacts in the collected data. The first disadvantage is certainly not remediable from the Earth’s surface, while the second disadvantage has previously been ameliorated by building in very remote places. But, as of the last decade, the construction of radio facilities in remote sites is no longer sufficient: more radio-loud technology has spread across the land and, recently, the emissions from large satellite constellations have severely impacted ground-based radio data \citep{di2023unintended}. Therefore, it is no longer a simple matter to find a quiet environment on Earth.

To combat the problem of RFI we can build telescope antennas in locations which are not in close proximity to each other. By using the techniques of Very Long  Baseline Interferometry (VLBI) we can then connect the telesocpes through software and correlate the signals. This means that any signal that is distinct to the local environment would not significantly impact the final result. There are approximately six VLBI networks on Earth used by the scientific community but in each case they offer micro-arcsecond scale resolution of the Universe but also suffer from a large computational challenge in more than the data for a single source is correlated and imaged. %CDT: I am trying to get some concrete numbers to put here.

However, this limits the potential of contamination from RFI but does not eliminate it as there is still a potential for sattalites to be observable in the field-of-view of more than one antenna simultaneously. This also does not change the limits on the frequency range observable from Earth.  

To combat both of these problems in other wavelengths, we now readily build telescopes on satellites, airplanes, and even the International Space Station. However, this solution ironically contributes to the problem of anthropogenic interference for radio facilities on the ground. Due to the increasing privatization and decreasing costs of space flight and launch equipment, we can now consider performing radio astronomy from non-terrestrial environments as well.


%For the low frequencies, there is no alternative than getting above the atmosphere, which will be discussed in the next session.  For meter and centimeter wavelengths one could use very long separations between coherently connected antennas.  But 

\subsection{Low Earth Orbit}
Getting into earth orbit is beneficial for low frequency observations, although there will be some leakage.  Additionally, given the size scale of low frequency antennas this is not a unsurmountable problem. Since the 1970s, several astronomy-based observatories have successfully launched into low-Earth orbits to provide a unique views of the skies. Some of the most sucessful missions provided decades of legacy data which conitnues to contribute to new discoveries. Some examples include the \textit{Hubble Space Telescope}, \textit{Hershel Space Telescope}, and \textit{Chandra}. Many of these were developed due a need to leave the atmosphere of the Earth to effectively study the cosmos at their associated wavelenths (i.e. optical, infrared, and x-ray). 

Even though an orbiting radio astronomy telescope provides the advantage of escapting the ionosphere it does not create an barrier against anthroprogenic radio signals. Therefore, although it increases the scientific scope over ground-based systems, additional algorithms are still needed to sort through and remove extraneous radio data not associated with astronomical objects.

\subsection{Lunar Orbit}
Although there are some proposed lunar orbit telescopes, both Earth orbit and Lunar Orbit telescopes suffer from the need to be light-weight in design and function. For example the proposed Dark Ages Polarimeter Pathfinder (DAPPER) project is a ``smallSAT" design with a narrow scientific focus of studying the cosmic dark ages and operated across 17--38\,MHz. Creating a telescopes that orbits the lunar surface offers the advantage of RFI shield from the moon during part of the cycle, a high rate of recharging through the solar panels for when the sattellite is awary from the sun, and removes the wavelength restrictions set by the Earth's atmosphere. 

To create a telescope with a broad scientific focus and, therefore, significant wavelenthg observing range, a stationary telescope on the lunars far-side avoides many of these limitations.