This paper presents a science case for an instrument landing and operating from the lunar farside surface.  Other possible missions obviously exist, which will be briefly discussed in the following.
However, we believe that a lunar farside surface instrument possesses compelling and unique aspects such as a consistently radio-quiet environment, site stability (e.g., \citealt{2021arXiv210305085B}), study operational requirements when operating on the moon, and being able to design a telescope with broad frequency capabilities, all contributing to a larger understanding important for designing larger missions in the future. 
And, as mentioned earlier, this mission is not predicated on what might be done if one had a fairly large budget to design and build a radio telescope on Earth, but rather what could one do for an inexpensive lunar mission that advances science and our understanding of operating on the moon.
%CDT: Suggest moving this all to the indroduction

\subsection{Surface of Earth}
%The surface of the Earth obviously presents the uniquely beneficial property of ease of access. This ease of access of course also applies to nearly all of human activity that know produces radio frequency interference.  Additionally, the atmosphere impacts the transmission of radio waves at very low frequencies (below around 20 MHz) and high frequencies (above around 20 GHz, which generally don't suffer RFI issues).

Although radio telescopes are often built in remote places away from dense urban areas, they are still significantly more accessible than space-based instruments. This proximity allows for large apertures, maintainable and changeable hardware, and the ability to process and transport large amounts of data. However, the disadvantages are two-fold, severe, and unmitigatable: 1) the atmosphere and ionosphere block and distort signals at many important radio frequencies of interest and 2) on- and off-world human activities cause significant radio frequency interference (RFI) that contaminates the collected data with erroneous artifacts. The first disadvantage is certainly not remediable from the Earth’s surface, while the second disadvantage has previously been ameliorated by building in very remote places. But in recent decades, the construction of radio facilities in remote sites is no longer sufficient: more radio-loud technology has spread throughout the country and, recently, emissions from large satellite constellations have severely affected ground-based radio data \citep{2025arXiv250410032Z, 2025arXiv250602831G, 2025rfic.confE..46Z, 2024A&A...689L..10B}. Therefore, it is no longer a simple matter to find a quiet environment on Earth -- all locations imply a loss of discovery opportunity.

To combat the problem of RFI we can build telescope antennas in locations which are not in close proximity to each other. Using the very long baseline interferometry techniques (VLBI), we can then connect the telescopes through software and correlate the signals. This means that any signal that originates in the local environment would not significantly impact the final result. There are approximately six VLBI networks on Earth that are used by the scientific community. While they each offer microarcsecond scale resolution of the Universe, they suffer from significant computational challenges when correlating data from multiple sources and wide fields. %CDT: I am trying to get some concrete numbers to put here.
The use of VLBI limits RFI contamination but does not eliminate it because RFI-producing satellites can appear simultaneously within the field of view of multiple antennas. VLBI also does not change the limits of the frequency range observable from Earth because of atmospheric and ionospheric effects.
However, this technique remains our best long-term path forward to deploy significant sensitivity, even with the negative discovery impact of RFI.

To combat both of these problems in other wavelengths, we now readily build telescopes on satellites, airplanes, and even the International Space Station. However, this solution ironically contributes to the problem of anthropogenic interference for radio facilities on the ground. Due to the increasing privatization and decreasing costs of space flight and launch equipment, we can now consider performing radio astronomy from non-terrestrial environments as well.  However, as pointed out, the same technology is driving other activity to space as well.


%For the low frequencies, there is no alternative to getting above the atmosphere, which will be discussed in the next session.  For meter and centimeter wavelengths one could use very long separations between coherently connected antennas.  But 

\subsection{Low Earth Orbit}
Getting into Earth orbit is beneficial for low-frequency observations, although there will be some leakage.
%Given the size scale of low-frequency antennas, this is, unfortunately??, not an unsurmountable problem. 
Since the 1970s, several space-based observatories have been successfully launched into low-Earth orbits to provide unique views of the skies. Some of the most successful missions provided decades of legacy data that continue to contribute to new discoveries. Some examples include the \textit{Hubble Space Telescope}, \textit{Hershel Space Telescope}, and the \textit{Chandra X-ray Observatory}. These missions were developed because of the need to leave the atmosphere of the Earth to effectively study the cosmos at their associated wavelengths (i.e. optical, infrared, and x-ray). 

Although an orbiting radio astronomy telescope provides the advantage of escaping the ionosphere, increasing its scientific scope over ground-based systems, it does not create an effective barrier against anthropogenic radio signals. Although algorithms can help identify and flag some RFI, they cannot fully restore data quality.  Contaminated data are often irretrievably lost, and subtle astronomical signals can be completely masked by this interference.  So, for VHF and above, there is no advantage of low Earth orbit over Earth-based sites, and orbit provide a multitude of additional challenges.

\subsection{Lunar Orbit}
Lunar orbit has some advantages: no risk of landing, no need for a relay satellite, and a decreased need for batteries.  The preference is a matter of mission goals, and lunar orbit has some key disadvantages:
\begin{itemize}
    \item potential of RFI due to necessary station-keeping activities,
    \item rapidly changing platform characteristics,
    \item RFI-quiet only for roughly 1/3 of the time,
    \item we don't learn about the Moon itself,
    \item we don't learn about operating on the Moon.
\end{itemize}
Note that the communication relay orbiters are already going to be in place by the time of LFT3 so the communications infrastructure is available, and the spacecraft will not require the higher-power and higher-frequency communications package to communicate back to Earth.

%Although there are some proposed lunar orbiting telescopes, both Earth-orbiting and lunar-orbiting telescopes suffer from the need to be lightweight in design and function. For example, the proposed Dark Ages Polarimeter Pathfinder (DAPPER) project is a "smallSAT" design with a narrow scientific focus of studying the Cosmic Dark Ages and operated across 17--38\,MHz. Creating a telescope that orbits the lunar surface offers the advantage of RFI shield from the Moon during part of the cycle, a high rate of recharging through the solar panels for when the satellite is facing the sun, and removes the wavelength restrictions imposed by the Earth's atmosphere. 

\subsection{Lunar Surface}
%To create a telescope with a broad scientific focus and, therefore, significant wavelength observing range, a stationary telescope on the lunar farside avoids many limitations of orbital telescopes.  
After landing, a farside surface telescope would be entirely stationary with respect to the Moon, simplifying its operations and making them more similar to those of a traditional Earth-based telescope than an orbital telescope.  As explained before, being on the lunar farside also guarantees continuous shielding from RFI rather than the only intermittent shielding a lunar orbiting telescope would have.  Lunar orbiting telescopes face an unfortunate tradeoff between RFI shielding and orbital stationkeeping complexity. Lower lunar orbits spend more of their time in the lunar farside's radio shadow, shielding them from RFI, but decay much more quickly due to lunar mass concentrations that destabilize orbits. Higher orbits can last much longer without decaying but are accordingly shielded for much shorter periods.  A surface-based farside telescope would escape this trade-off entirely, as it eliminates all orbital stationkeeping requirements and would be shielded against RFI 100\% of the time.  Orbital station-keeping also has the potential to produce self-generated RFI.  The dwell time on a patch of sky is quite long (about 30 times that of Earth), which affords better survey properties.

