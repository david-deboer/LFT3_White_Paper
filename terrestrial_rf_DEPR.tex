As is well-known, observing from the surface of the Earth is heavily impacted by radio frequency interference, or RFI\footnote{Note that this uses the term ``RFI'' as any radio frequency energy impacting radio astronomy observations regardless of frequency and not limited to assigned radio astronomy radio bands.}.  The way this is usually handled in radio astronomy is to try to use relatively clear bands and flag out frequencies and times impacted by RFI.  Some effort has been made to ``look through'' RFI, but improvements have usually been limited to 30 -- 40 dB. Using widely separated antennas also provides a large degree of RFI suppression, since the RF power due to local interference is not correlated.

\subsection{Very Long Baseline RF}
In order to understand the degree of RFI suppression, a simple model has been developed to determined idealized RF suppression.  The model uses band-limited correlated sky noise, band-limited uncorrelated system noise, a correlated sky tone and uncorrelated RFI.  The system parameters are similar to the Breakthrough Listen processing pipeline 3 MHz coarse channel with 1.23 Hz resolution.  The sky tone represents a $10^{12}$ W EIRP at 100 ly.  The RFI is a 1 kW FM signal at 50 km.  Figure \ref{fig:rfi-site} shows the tone and RFI as received by a 12m antenna.